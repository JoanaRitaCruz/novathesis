%!TEX root = ../template.tex
%%%%%%%%%%%%%%%%%%%%%%%%%%%%%%%%%%%%%%%%%%%%%%%%%%%%%%%%%%%%%%%%%%%
%% chapter1.tex
%% NOVA thesis document file
%%
%% Chapter with introduction
%%%%%%%%%%%%%%%%%%%%%%%%%%%%%%%%%%%%%%%%%%%%%%%%%%%%%%%%%%%%%%%%%%%

\typeout{NT FILE chapter1.tex}%

\chapter{Introduction}
\label{cha:introduction}

\prependtographicspath{{Chapters/Figures/Covers/}}

% epigraph configuration
\epigraphfontsize{\small\itshape}
\setlength\epigraphwidth{12.5cm}
\setlength\epigraphrule{0pt}

Storytelling has always been part of the work of Museums~\cite{bedford_storytelling_2001} but due to the changes in 
the way people consume information and their relationship with knowledge ~\cite{wyman_digital_2011} along with the ever-evolving digital landscape,
these institutions are increasingly expanding their focus from the display and collecting of artifacts to the goal of being dynamic, interactive spaces
where the coexistance of virtual information and tangible artifacts and spaces create an interactive educational and cultural experience.

Storytelling has always been a core concept in the work of Museums~\cite{bedford_storytelling_2001} but the technological evolution and changes in visitor's abilities and interests
have urged museums to expand their focus, no longer being just places of display, collection and storage of artifacts but increasingly becoming
dynamic, interactive spaces where virtual information coexists with tangible artifacts.



\section{Motivation}
\label{sec:motivation}

Museums exist because someone believed there was a story worth telling for generations to come, the stories collected by these institutions
preserve individual and collective memories~\cite{bedford_storytelling_2001} and incite opportunities of learning and reconnecting with the past.
However in this technological age traditional exhibits are becoming insufficient ~\cite[pp. 326]{seale_education_2023}, these type of experience is no longer engaging to audiences who 
are accostumed to personalized experiences and quick information gathering that is enabled by the touch of a screen.\\
Therefor the mission of Museums and Cultural Heritage sites is changing, from places of collecting, display and storage to becoming
dynamic and interactive spaces, where the coexistance of virtual information and characters with tangible artifacts and spaces leads to a memorable and personalized experience ~\cite[pp. 3978]{dal_falco_museum_2017}.

With the expansion of the digital world and the evolution of Virtual and Mixed Reality technologies, interactive cultural experiences powered by
interactive displays, mobile aplications, forms of digital storytelling and gamification strategies are in high demand~\cite[pp. 105]{danks_interactive_2007}, and many locations
have sought the development of Virtual tours or experiences specifically tailored to enhance the engagement in their exhibits. A lot of these projects
are either fully virtual, or are not easily transposed to other museums, given they are developed for a specific institution or historical site.
Although virtual reality allows for a much more immersive experience, the tangible artifacts and the feeling of seeing something in person still have their value,
and Augmented or Mixed Reality technologies in their current state can provide a hybrid experience that encompasses both the advantages of digital overlays
and the on-site exhibits with real artifacts.

As already mentioned museums are, in their core storytellers, and the collected objects themselves are entities that represent and tell a story,
however for modern audiences, a more direct and human-like storytelling approach based on characters has proved rather effective~\cite{bekele_survey_2018},
whether it be virtual guides, historical or fictional character, having a narrator for the story who embodies the storyteller is usually well received,
after all humans are also storytellers and with the rise of social media and the increasing of time spent on such apps, people have started craving 
experiences of or that emulate human interaction.

Regarding the search for personalized and interactive moments, this need reflects a shift in the way people find and consume information, ~\cite{wyman_digital_2011}
whereas before the search of information was part of the process and the journey was in itself an interaction, now people have all the information they want
on their fingers. Before the 



\section{Challenges and objectives}
\label{sec:challenges_and_objectives}

Did you learn how to drive by sitting by the wheel and throwing your car into the road?  Most probably you did take your time \emph{learning the rules} and \emph{practicing} first! Likewise, it is not wise to throw yourself at the task of writing a thesis/dissertation in \LaTeX\ without seriously considering the following \ntindex{recommendation}!

\begin{tcolorbox}[colback=green!8]
  If you are going to spend zillions of hours writing your thesis/dissertation using the \gls{novathesis} \LaTeX\ template (or some other \LaTeX\ template), be wise and spend a couple of hours learning how to use it properly by reading its manual.  And then, be even wiser, and spend a few more hours \href{https://github.com/joaomlourenco/novathesis/wiki\#learning-latex}{learning some \LaTeX}.  I am sure that the time you are investing now will pay itself countless times before you submit your thesis/dissertation.\\\parbox{\linewidth}{\raggedleft---~\emph{João Lourenço}}
\end{tcolorbox}

\section{Solution}
\label{sec:solution}

\ntindex[Recognition]{}

The \gls{novathesis} template was born in~1996, and what you see now accumulates to many many hundreds (thousands?!) of working hours, unpaid and stolen from family and friends.  This work is available to the community under the \href{LaTeX project public license}{\LaTeX\ Project Public License v1.3c}, which means you are entitled to use it for free and change it at your will.  However, if you decide to use this template to write your thesis/dissertation, \textbf{be fair to the developers} and:
\begin{enumerate}
  \item \ntindex[novathesis!Citation]{} Cite the \gls{novathesis} manual~\cite{novathesis-manual} in a place of your choice (e.g., in the \emph{Acknowledgments}) of your thesis/dissertation with “\verb!\cite{novathesis-manual}!” .  If you cite it this way, the correct entry will be added automatically to your bibliography (no need to worry with the necessary BibTeX entry, as it will be added automatically);
  \item Go to the
\href{https://github.com/joaomlourenco/novathesis}{\ntindex[GitHub!project web page]{project web page} in GitHub} and give the project a \ntindex[GitHub!stars]{star} (marked with a red ellipse at the top-right in Figure~\ref{fig:github}); and
  \item Make a \ntindex[donations]{donation} by visiting the \gls{novathesis} project page and clicking in the button marked with a green ellipse at the top-center in Figure~\ref{fig:github}).  Alternatively, just click \href{https://www.paypal.com/donate/?hosted_button_id=8WA8FRVMB78W8}{\fcolorbox{DarkGreen}{gray!15}{\textbf{~HERE~}}} and your browser will be directed to the right page.
\end{enumerate}

\begin{figure}[htbp]
    \centering
    \includegraphics[width=0.5\linewidth]{github1}
    \caption{The \gls{novathesis} project web page in GitHub.}
    \label{fig:github}
\end{figure}

\section{Contributions}
\label{sec:contributions}
