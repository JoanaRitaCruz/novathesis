%!TEX root = ../template.tex
%%%%%%%%%%%%%%%%%%%%%%%%%%%%%%%%%%%%%%%%%%%%%%%%%%%%%%%%%%%%%%%%%%%
%% chapter1.tex
%% NOVA thesis document file
%%
%% Chapter with introduction
%%%%%%%%%%%%%%%%%%%%%%%%%%%%%%%%%%%%%%%%%%%%%%%%%%%%%%%%%%%%%%%%%%%

\typeout{NT FILE chapter1.tex}%

\chapter{Introduction}
\label{cha:introduction}

\prependtographicspath{{Chapters/Figures/Covers/}}

% epigraph configuration
\epigraphfontsize{\small\itshape}
\setlength\epigraphwidth{12.5cm}
\setlength\epigraphrule{0pt}

Museums exist because someone believed there was a story worth telling for generations to come, the stories collected by these institutions
preserve individual and collective memories~\cite{bedford_storytelling_2001} and incite opportunities of learning and reconnecting with the past.
However in this technological age traditional exhibits are becoming insufficient ~\cite[pp. 326]{seale_education_2023}, these type of experience is no longer engaging to audiences who 
are accustomed to personalized ones and quick information gathering that is achieved by the mere touch of a screen.\\
Therefor the mission of Museums and Cultural Heritage sites is changing, from places of collecting, display and storage to becoming
dynamic and interactive spaces, where the coexistence of virtual information and tangible artifacts and spaces leads to a memorable and personalized experience ~\cite[pp. 3978]{dal_falco_museum_2017}.


\section{Motivation}
\label{sec:motivation}

With the expansion of the digital world and the evolution of Virtual and Mixed Reality technologies, interactive cultural experiences powered by
interactive displays, mobile applications, forms of digital storytelling and gamification strategies are in high demand~\cite[pp. 105]{danks_interactive_2007}, and many locations
have sought the development of Virtual tours or experiences specifically tailored to enhance the engagement in their exhibits. A lot of these projects
are either fully virtual, or are not easily transposed to other museums, given they are developed for a specific institution or historical site,
and so there is a lack of adaptable and scalable solutions in this area.

Although virtual reality allows for a much more immersive experience, the tangible artifacts and the feeling of seeing something in person still hold value,
and Augmented or Mixed Reality technologies in their current state can provide a hybrid experience that encompasses both the advantages of digital overlays
and of on-site exhibits with real artifacts, without compromising the state of preservation of such objects whilst still providing visitors with a closer
and more detailed look. Artifacts in recovery can also be filled in digitally, reducing the impact of such pieces missing.But it is also important to note
that not all relevant information about the collected objects is easily displayed in an engaging way, some aspects like the materials of a piece,
the year of it's making or even facts about the creator are important but are badly received by visitors after they have seen several similar 
descriptions and details. Nonetheless, this type of approach, centered in Augmented reality, enriches the physical visits without distracting from
the authenticity of seeing real, tangible artifacts. 

As already mentioned, museums are, in their core, storytellers and the collected objects themselves are entities that represent and tell a story,
however for modern audiences, a more direct and human-like storytelling approach based on characters has proved rather effective~\cite{bekele_survey_2018},
whether it be virtual guides, historical or fictional characters, having a narrator for the story who embodies the storyteller is usually well received,
after all humans are also storytellers and with the rise of social media and the increasing time spent on such apps, people have started craving 
experiences of or that emulate human interaction.

Regarding the search for personalized and interactive moments, this need reflects a shift in the way people find and consume information,
whereas before the search of information was part of the process and the journey was in itself an interaction, now people have all the information they want
on their fingers ~\cite{wyman_digital_2011} and expect it as quickly as possible. However the popularity of adventures such as escape rooms, the success of 
gamification in applications with a cultural background and the growth of choice-based video-games reveal an interest from the public in more user-centric
explorations. Instances in which the visitor's actions directly influence the course of their experience.

Taking all this information and context into account the design of an interactive and adaptable museum experience for institutions such as \textit{Museu Nacional dos Coches (MNC)} 
or \textit{Museu Nacional do Traje (MNT)}, among others, can be achieved by compiling character-driven storytelling with gamification techniques
whilst addressing the evolving expectations of modern audiences by giving the visitors control over their experience, in carefully selected choice moments.
By leveraging different forms of interaction and blending digital and physical elements, such system can bridge the gap between traditional exhibits and the dynamic digital experiences, ensuring
the continuous relevance of cultural heritage. 

\section{Challenges and objectives}
\label{sec:challenges_and_objectives}
The main objective of this dissertation, following up to it's motivations, is the design of a storytelling experience in the context of
museums and other cultural and historical sites that is adaptable to different locations. This dissertation also aims at developing an approach that
complements the on-site experience with digital characters, digitized objects and information overlays without detracting from the tangible
artifacts and physical characteristics of the site or exhibition and with a large focus on the narrative.

This work encompasses many challenges and is guided by some research questions, the first one, mentioned in \cite[p. 463]{wyman_digital_2011} being:

\subsection*{\textbf{Q1} -- \textit{How can a museum best frame content to make it desirable?}}

The answer to this question in this case is by making the content part of an overall story, presenting it with interactive characters and
crafting an user-centric experience where the actions of visitors define their experience.

\subsection*{\textbf{Q2} -- \textit{Can character-based interactive storytelling serve the purpose of museums and how?}}

By choosing characters that blend in with the exhibition, per example a coachman in the \textit{Museu Nacional dos Coches}, 
and allowing the consumers to make questions and interacting with this character a new perspective can be presented in the museum,
and the transmission of information can be more dynamic and more effective.

\subsection*{\textbf{Q3} -- \textit{Can personalized experiences and variable narratives encourage repeat visits?}}

If the story changes based on the actions of the users, such as the choices made or questions asked, the experience,
even for the same route in the museum or historical site, will always be slightly different from visit to visit. Also with a character-based approach
other things can be explored such as different views for the same route, different characters. And the experience may also encompass several routes that
group related points of interest and this routes can be explored across multiple visits.

\subsection*{\textbf{Q4} -- \textit{ In what ways can museums use gamification to transform traditionally tedious or detailed information into engaging educational content?}}

The definition of games that depend on or transform this information may prove effective. The main idea is that if the user sees this information they are rewarded
as if in a game, per example if they see the material of a piece they unlock this material to use in a later game or get some sort of digital achievement.

\subsection*{\textbf{Q5} -- \textit{ How can pieces that are being recovered or are unavailable at the moment be digitally incorporated?}}

Even if an artifact is not physically present if the digitized 3D model is available it can be displayed, zoomed in and rotated by the user, diminishing
the impact of the object missing.

\subsection*{\textbf{Q6} -- \textit{ Can the lack of immersion of Mobile Augmented Reality be counter-balanced by it's availability and reduced price?}}

Although head-mounted displays and virtual reality prove more immersive, everyone now-a-days owns a smart phone, and the AR capabilities of such mobile devices
have come a long way, it is also a lot more affordable for the institutions themselves to incorporate a mobile app than acquire head-mounted displays or 
virtual reality hardware devices. Moreover there is not so much concern of spacial restrictions that could prevent several visitors from
performing the same digital route.

\subsection*{\textbf{Q7} -- \textit{ How can such experience be made easily adaptable to different museums and historical sites?}}

To make the experience easily adaptable a platform has to be developed, one that encompasses the steps of the created approach in a way that is low-code,
accessible and that allows information, whether it be text, specific characters, specific interactions or personalized mini-games to be easily uploaded
and  integrated.\\


Along with the main objective this dissertation aims at addressing all these challenges, trying out and researching upon the purposed solutions
and ultimately, this work strives to develop a platform, create an approach and design a museum exhibit experience that encompasses several routes,
with interactive characters and interesting information overlays. This experience shall also take advantage of gamification techniques, include mini-games
and explore the creation of non-linear narratives based on user choices and actions, in order to enrich the on-site museum experience.

\section{Solution}
\label{sec:solution}
This thesis proposes a solution for an easily adaptable on-site museum experience, powered by character-based digital storytelling, gamification and that 
encompasses a high focus on narrative generation and the enhancement of the physical museum tour. This solution takes advantage of the museums storyteller
nature and transposes it into the form of historically inspired and interactive non-player characters in an attempt to create a highly engaging adventure.

The way the experience is planned to take place is by the integration of several Points Of Interest (POIs) in the institutions physical spaces, each of this POIs will
have associated characters and a main character can be selected to lead the action. The interaction with the characters is made through questions,
some predefined questions that are chosen in the context of the overall story, and some free question. The questions asked and their different variations
alter the resulting narrative.

Regarding the routes, each route groups related POIs, the points relation can be historical, temporal or simply spacial and routes can be of three types:
unordered and unrelated routes, where the points don't encompass an overall story; related but unordered, the POIs belong to the same story but can be visited
by any order; ordered and related routes, the points belong to the same story and must be visited in a certain order.

At the end of each visit to a POI a part of the story is generated, at the end of a route a full story, influenced by the questions asked and the order in which the points
are visited, is revealed.

The experience also includes two base mini-games that address the challenges (mentioned in section \ref{sec:challenges_and_objectives}) of displaying traditionally tedious information
in an engaging way and how to incorporate objects that are unavailable:
\begin{itemize}
  \item For the first challenge, a dress-up like game will be included at the end of a route, when the more detailed information is visited in a POI a related
  texture or piece is unlocked to use in the game. For the MNC the visitor can build and customize their own coach 3D model, based on the unlocked textures and subsequently
  the visited information; for the MNT it would be an actual dress-up game with clothing pieces and fabrics to be unlocked; for a traditional art museum this mini-game
  could be adapted to some sort of digital canvas with different materials and types of ink.
  \item Regarding the unavailable object a 3D puzzle will appear during or at the end of a POI dialogue, in this puzzle the digitized 3D model of the artifact is broken 
  into pieces either cubical or corresponding to parts of the object, when the user puts the pieces back together the 3D model is unlocked and the user can now handle and
  inspect the missing relic.
\end{itemize}

In order to make the solution adaptable and scalable, these steps are to be encompassed and automated by a platform to be developed. In the platform
the historical information and narrative and the digitized 3D models are to be received as input, the user then also must create and allocate the characters to each Point Of Interest
as well as define the routes and it's nature. New mini-games must and might also be added as well as the noting of unavailable objects and it's allocation to specific POIs,
each point also must hold information of what type of rewards visiting it or seeing certain parts of it provide.


\section{Contributions}
\label{sec:contributions}

The expected contributions of this work, apart from this proposal of thesis and the dissertation itself, are:
\begin{itemize}
  \item The platform and approach developed for an adaptable and scalable on-site, storytelling driven museum experience;
  \item The mobile application that accompanies the Augmented Reality visit.
  \item The 3D puzzle mini-game and the system that divides a 3D model into pieces;
  \item The dress-up like mini-game;
  \item Prototypes that compare: AR foundation and the Vuforia plugin;
  \item Experiments with Narrative Procedural Generation as well as graph a Graph story-node system for the generation of the resulting story.
  \item Experiences with Knowledge-based or AI solutions for the interactive characters
  and the answers to the free questions.
  \item The result analysis through user-testing and questionnaires such as: System Usability Scale; User Experience Questionnaire;
   Presence Questionnaire and the Museum Experience Scale.

\section{Document structure}
\label{sec:doc_structure}
